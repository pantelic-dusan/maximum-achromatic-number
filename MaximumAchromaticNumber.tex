
% !TEX encoding = UTF-8 Unicode
\documentclass[a4paper]{article}

\setcounter{tocdepth}{1}

\usepackage{color}
\usepackage{url}
\usepackage[T2A]{fontenc} % enable Cyrillic fonts
\usepackage[utf8]{inputenc} % make weird characters work
\usepackage{graphicx}

\usepackage[english,serbianc]{babel}

\usepackage[unicode]{hyperref}
\hypersetup{colorlinks,citecolor=green,filecolor=green,linkcolor=blue,urlcolor=blue}

\usepackage{listings}

\definecolor{mygreen}{rgb}{0,0.6,0}
\definecolor{mygray}{rgb}{0.5,0.5,0.5}
\definecolor{mymauve}{rgb}{0.58,0,0.82}

\lstset{ 
  backgroundcolor=\color{white},
  basicstyle=\scriptsize\ttfamily,  
  breakatwhitespace=false,         
  breaklines=true,                 
  captionpos=b,                   
  commentstyle=\color{mygreen},   
  deletekeywords={...},           
  escapeinside={\%*}{*)},          
  extendedchars=true,              
  firstnumber=1000,                
  frame=single,	                   
  keepspaces=true,                 
  keywordstyle=\color{blue},       
  language=Python,                 
  morekeywords={*,...},            
  numbers=left,                    
  numbersep=5pt,                   
  numberstyle=\tiny\color{mygray},
  rulecolor=\color{black},        
  showspaces=false,               
  showstringspaces=false,          
  showtabs=false,                  
  stepnumber=2,                   
  stringstyle=\color{mymauve},     
  tabsize=2,	                   
  title=\lstname                   
}

\begin{document}

\title{\Large Решавање проблема максималног ахроматског броја\\ \small{Научни рад у оквиру курса Рачунарска интелигенција\\ Математички факултет}}

\author{Душан Пантелић\\ pantelic.dusan@protonmail.com}

\maketitle

\begin{center}
	\includegraphics[width=3cm]{images/pmf.png}
\end{center}

\abstract{
Кроз овај рад читалац ће бити упућен у проблем максималног ахроматског броја, као и у решевање истог. Пре свега, шта представља максимални ахроматски број, његов значај, примену, али и проблеме при његовом одређивању. Представљени су алгоритми за решавање проблема и њихова компарација, као и њихове позитивне и негативне стране. Читалац ће такође добити увид у експерименталне податке извршавања представљених алгоритама са примењеним статистичким методама зарад стварања шире слике о практичној примени. На крају рада сумирани су закључци истраживања и изложени правци за даље истраживање и унапређивање.
\tableofcontents

\newpage

\section{Увод}
\label{sec:intro}

\section{Решење проблема}
\label{sec:problem-solution}

\section{Експериментални резултати}
\label{sec:experimet-results}

\section{Закључак}
\label{sec:conclusion}


\addcontentsline{toc}{section}{Литература}
\appendix
\bibliography{maximum-achromatic-number} 
\bibliographystyle{plain}

\newpage
\end{document}